\documentclass[letterpaper,11pt]{article}
\usepackage[margin=20mm]{geometry}
\usepackage{setspace} \doublespacing
% \usepackage{multicol}
\newcommand\T{\rule{0pt}{10ex}}       % Top strut
\newcommand\Bshort{\rule[-5ex]{0pt}{0pt}} % Bottom strut
\newcommand\B{\rule[-10ex]{0pt}{0pt}} % Bottom strut
\newcommand\Blong{\rule[-40ex]{0pt}{0pt}} % Bottom strut
\newcommand\textlcsc[1]{\textsc{\MakeLowercase{#1}}}
\usepackage{booktabs}% http://ctan.org/pkg/booktabs
\newcommand{\tabitem}{~~\llap{\textbullet}~~}
\usepackage{ amssymb }
\thispagestyle{empty}

\begin{document}

% \title{Learning-focused lesson plan}
% \maketitle
%% \begin{center}
%%   \Large \textlcsc{Learning-focused lesson plan}
%% \end{center}

\vspace{5mm}

%% meta-data
\begin{tabular}{p{0.4\linewidth} p{0.5\linewidth}}
  Instructor: AG Schissler & Course: STAT 445-645 \\
  Unit title: Intro to R & Lesson title: Intro to the course \\
  Date(s): 27 Aug 2018 & Lesson length: 75 min \\
\end{tabular}

%% type of lesson
\vspace{5mm}
Type of lesson:~ $\boxtimes$ Acquisition \hspace{5mm} $\square$ Extending/refining  \hspace{5mm} $\square$ Assessment

\vspace{5mm}

%% lesson plan graphic organizer
\begin{tabular}{|p{0.3\linewidth} | p{0.7\linewidth}|}
  \hline
  \textbf{Unit Learning Outcomes} & Students will be able to (SWBAT) use R to store/manipulate data and create reproducible documents in R Markdown. \\
%%   \hline
%%   Key vocabulary & \\
  \hline
  \textbf{Beginning of class} & 5-10 min, use ``primary'' effect
  \\
  \hline
  Engagement trigger: 
  \newline Most important information: 
  \newline Start-of-class-work:
                                  & Interactive R demonstration; rationale for learning/personal story
                                    \newline Show website, syllabus, learning outcomes, rubrics
                                    \newline open R Studio and navigate to course website
  \\
  \hline
  \textbf{Middle of class} & 10-20 min segments, follow ``I do'', ``We do'', ``You do''
  \\
  \hline
  Segment 1 learning outcomes: 
  \newline Strategies \B:

                                  & SWBAT use course resources and basic git commands
                                    \newline Instructor leads tour of website, Canvas, forum, github
                                    \newline Students use git and web browser to retrieve R demonstration
  \\
  Segment 2 learning outcomes: 
  \newline Strategies \B:
                                  & SWBAT use R Markdown to produce HTML, PDF, or Word documents
                                    \newline Instructor demonstrates/discusses R Markdown in RStudio
                                    \newline Students compile (``knit'') an R Markdown document
  \\
  Segment 3 learning outcomes: 
  \newline Strategies \B:
                                  & SWBAT use DataCamp for online, independent learning
                                    \newline Instructor tours DataCamp and expectations
                                    \newline Students enroll, tour, play for a bit
  \\
  Segment 4 learning outcomes:
  \newline Strategies \B:
                                  &
  \\
  \hline
  \textbf{End of class} & 5-10 min, ``Recency'' effect, Student-led summary, balance/vary activities \\
  \hline
  \Bshort
                                  & Answer the following ``What?'', ``So what?'', ``Now what?''
  \\
  \hline
\end{tabular}


\end{document}