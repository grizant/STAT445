%A simple LaTeX syllabus template by Luke Smith.
%I believe this is a variation of a template I got by Mike Hammond...?

%\documentclass[11pt,twocolumn]{article}
\documentclass[11pt,onecolumn]{article}
\usepackage{hyperref}
\usepackage[margin=0.75in]{geometry}
\usepackage{titlesec}
\usepackage{longtable}
\usepackage{gb4e}
%\usepackage[margin=.75in]{geometry}
\usepackage{textcomp}
\pagenumbering{gobble}
%% \usepackage[utf8]{inputenc}
%% \usepackage[english]{babel}
%% \usepackage[usenames, dvipsnames]{color}

%paragraph formatting
\setlength{\parindent}{0pt}
\setlength{\parskip}{3pt}

\usepackage{url}

\hypersetup{
    colorlinks=true,
    linkcolor=blue,
    filecolor=magenta,      
    urlcolor=blue,
}

\titlespacing\subsection{0in}{\parskip}{\parskip}
\titlespacing\section{0in}{\parskip}{\parskip}

%Just fill these in to fill in the basic syllabus information.
\newcommand{\coursename}{Introduction to Statistical Computing - STAT 445/645}
\newcommand{\semester}{Fall 2018}
\newcommand{\roomnumb}{DMSC 106}
\newcommand{\classtimes}{Mon,Wed 2:30pm - 3:45pm}
\newcommand{\myname}{A.~Grant Schissler}
%\newcommand{\myemail}{\href{mailto:aschissler@unr.edu}{aschissler@unr.edu}}
\newcommand{\myemail}{aschissler@unr.edu}
\newcommand{\office}{DMSC 224}
\newcommand{\officehours}{Tue 2:30pm-3:30pm, Wed 1:30pm-2:20pm, or by appointment}
% \newcommand{\university}{The University of Nevada, Reno}
\renewcommand{\familydefault}{\sfdefault}

\title{\textbf{\coursename}}
% \author{{\university}---{\semester}---{\roomnumb}---{\classtimes}}
\author{{\semester}---{\roomnumb}---{\classtimes}}
\date{}


\begin{document}
\maketitle

\vspace{-0.25in}
\noindent\makebox[\linewidth]{\rule{\textwidth}{1pt}}

\begin{center}
\begin{tabular}{llll}
\textbf{Instructor}:&\myname & \textbf{Contact}:&\href{mailto:\myemail}{\myemail}, 775-784-4661 (office)\\
\textbf{Office}:&\office & \textbf{Hours}:&\officehours\\
\end{tabular}
\end{center}

Computational skills have become an invaluable asset in our increasingly quantitative world. This course introduces students to the foundational skills and concepts needed in modern statistical computing, with an emphasis on high-level programming languages and statistical applications. Throughout, students will produce professional-grade, reproducible documents.

There are three main units to the course:~First, students learn to program in R, from basic data structures and flow control to advanced, efficient coding schemes (including \textit{parallelization}). Second, students learn data manipulation, exploration, and visualization skill. Third, students will learn to conduct statistical inference through simulation (Monte Carlo) techniques and optimization of probabilistic functions.

\section*{Catalog Description}
Introduction to statistical computing; data visualization and manipulation; document creation; graphics; simulation techniques; parallel computing; estimation; optimization; advanced statistical methods.

\section*{Course Pre-requisites}
STAT 352 or STAT 467/667 or with instructor approval.

\section*{400-level Student Learning Outcomes}
%% Student Learning Outcomes
%% Upon completion of this course, students will be able to:
%% 1. implement statistical simulation and resampling techniques, maximum likelihood estimation. 
%% 2. conduct a simulation-based power analysis. 
%% 3. write professional quality reports and computer code. 
\begin{description}
\item[UG1] Students will be able to implement statistical simulation, re-sampling techniques, and maximum likelihood estimation. 
\item[UG2] Students will be able to conduct a simulation-based power analysis. 
\item[UG3] Students will be able to write professional quality reports and computer code.
\end{description}

\section*{600-level Student Learning Outcomes}
\begin{description}
\item[GRAD1] Students will be able to use statistical computing methods to complete a research project and effectively communicate their findings.
\end{description}

\section*{Textbooks}
None required. Supplementary texts:
\begin{itemize}
  \itemsep0em
\item Rizzo, M. \emph{Statistical Computing With R}. 2008, Chapman \& Hall/CRC.
  %% $Textbook website:\\~\url{https://www.wiley.com/en-us/Introduction+to+Bayesian+Statistics%2C+3rd+Edition-p-9781118091562}
\item Efron, Bradley, and Trevor Hastie. \textit{Computer age statistical inference}. Vol. 5. Cambridge University Press, 2016.
  \item Grolemund, G., Wickham, H. \emph{R for Data Science}. 2016, O’Reilly.(Free at ~\url{http://r4ds.had.co.nz/})
\item The R Cookbook, by Paul Teetor
\end{itemize}

\section*{Online resources}
\begin{itemize}
    \itemsep0em
\item Course website:~\url{http:www.grantschissler.com/teaching/FA18/STAT445}, includes a working course schedule and approximate due dates of assignments and exams.
\item Course github repo:~\url{http:github.com/grizant/STAT445}
  \item DataCamp for online modules:~\url{https://www.datacamp.com/home}
\item Students are responsible for checking WebCampus (\url{http:wcl.unr.edu}) and their email accounts, and are assumed to be aware of all information posted to these sources prior to each meeting.
\end{itemize}

\section*{Assignments}
The instructor will assign online modules in DataCamp (approximately) weekly. Additionally, the instructor will assign roughly one ``lab'' assignment per week. These two types of assignments are categorized as ``Assignments''. Students will complete the online modules individually while lab assignments may be completed collaboratively. You are highly encouraged to discuss assignments between each other and with instructor. However, the works must be completed and submitted individually. All assignment, submissions, due dates, and feedback (grading) will be handled via WebCampus.

\section*{Exam policy} You will be allowed at one 8.5x11in page of handwritten (on both sides) notes for each midterm and three such pages for the final exam. If you believe that your grade for exam or assignment is incorrect, contact instructor at the office hours with a rational justification. All such requests must be submitted to instructor within one week after a grade is announced; late requests will not be granted. The final decision about new grade is made by the instructor. Please understand that everyone can make a mistake, and that mistakes can go both ways: higher or lower than original grade.

\section*{Midterms}
There will be two midterms, the first on Wednesday, October 3, and the second on Wednesday, November 7.

\section*{Final exam}
A comprehensive final examination will be held on Wednesday, December 19, 12:10pm - 2:10pm, DMS 106.

\section*{400/600 Students}
As indicated above, the student learning outcomes differ at the 400 and 600 levels. Students enrolled at the 600 level are required to complete a \textbf{term project} due the last day of class --- December 12. Graduate-level assignments include project milestones (in addition to the assignment's 400-level requirements).

\section*{Makeup, Late Policy}
Late online modules will receive a 50\% point deduction. Late ``lab'' assignments will not be graded. As a safety factor for emergencies, the instructor will drop your lowest two grades in the ``Assignments'' category. Late project work products (reports/presentations) will \textit{not} be graded. Lastly, there will be no early or make-up exams. However, if you need to miss an exam due to participation in official university activities (including athletics and other sanctioned activities), you must make arrangements with the instructor at least two weeks prior to the exam in question. Exceptions \textit{may} be made when a student misses work due to extraordinary situations, up to the instructor's discretion.

\newpage
\section*{Grading}
The final grades will be determined using the following percentages:

\begin{center}
\begin{tabular}{cc}
\begin{tabular}{l|c|r}	%For grade items (quizzes, homework, etc.)
Item&400-level& 600-level\\\hline\hline
  Assignments&40\% & 40\%\\
  Midterm Exams&40\%& 20\%\\
  Final Exam&20\%& 20\%\\
  Term project &-- &20\%\\
\end{tabular}
&
\begin{tabular}{ll}
A&90-100\\
B&80-89\\
C&70-79\\
D&60-69\\
F&59 or below
\end{tabular}
\end{tabular}
\end{center}

The instructor may deviate from the percentages. Letter grades with a +/- will likely be assigned for borderline cases (could be +/- 3\% points). 

\section*{Diversity Statement}
The University of Nevada, Reno is committed to providing a safe learning and work environment for all. If you believe you have experienced discrimination, sexual harassment, sexual assault, domestic/dating violence, or stalking, whether on or off campus, or need information related to immigration concerns, please contact the University’s Equal Opportunity \& Title IX Office at (775) 784-1547. Resources and interim measures are available to assist you. For more information, please visit \url{http:www.unr.edu/equal-opportunity-title-ix}.

\section*{Disability Statement}
Any student with a disability needing academic adjustments or accommodations is requested to speak with the \href{http:www.unr.edu/drc}{Disability Resource Center} as soon as possible to arrange for appropriate accommodations.

\section*{Academic Conduct}
No laptops, cell phones, mp3 players, or other electronics are to be used for personal reasons in class. If you are being disruptive during class you will be asked to leave. Disruptions in this context include inadequate participation. You must come to class on time and stay until the end of lecture. Tardy students will not be admitted to class. Please visit \url{http:www.unr.edu/student-conduct} for our official student code of conduct.

\section*{Academic Success Services}
Your student fees cover usage of the University Math Center, University Tutoring Center, and University Writing Center. These centers support your classroom learning; it is your responsibility to take advantage of their services. Keep in mind that seeking help outside of class is the sign of a responsible and successful student

\section*{University Recording Policy}
Surreptitious or covert videotaping of class or unauthorized audio recording of class is prohibited by law and by Board of Regents policy. This class may be videotaped or audio recorded only with the written permission of the instructor. In order to accommodate students with disabilities, some students may have been given permission to record class lectures and discussions. Therefore, students should understand that their comments during class may be recorded.

\section*{Academic Dishonesty}
Cheating, plagiarism, or otherwise obtaining grades under false pretenses constitutes academic dishonesty according to the code of this university. Academic dishonesty will not be tolerated and penalties can include canceling a student’s enrollment without a grade or giving an F for the assignment or for the entire course. See the University Academic Standards policy: \href{https://www.unr.edu/administrative-manual/6000-6999-curricula-teaching-research/instruction-research-procedures/6502-academic-standards}{UAM 6,502}.

\section*{Tentative course schedule}
\begin{center}
  \begin{tabular}{|c|c|c|c|}
    \hline
    Week & Unit & Monday & Wednesday \\
    \hline
    \hline
    1 & 1:~R Programming & Course structure/R/RStudio/RMarkdown/git & Data structures in R \\
    \hline
    2 & 1:~R Programming & \textbf{Labor Day (no class)} & Conditionals, flow \\
    \hline
    3 & 1:~R Programming & Loops, functions, apply & Lab \\
    \hline
    4 & 1:~R Programming & Writing functions (advanced) & Lab \\
    \hline
    5 & 1:~R Programming & Writing efficient R Code (e.g., \textsc{parallel}) & Lab  \\
    \hline
    6 & 1:~R Programming & Review session & \textbf{Midterm 1} \\
    \hline
    \hline
    7 & 2:~Working with data & Importing data & Lab \\
    \hline
    8 & 2:~Working with data & Cleaning data & Lab \\
    \hline
    9 & 2:~Working with data & Exploratory Data Analysis (EDA) & Lab \\
    \hline
    10 & 2:~Working with data & Data viz/grammar of graphics (\textsc{ggplot2}) & Lab \\
    \hline
    11 & 2:~Working with data & Review session & \textbf{Midterm 2} \\
    \hline
    \hline
    12 & 3:~Statistical inference & \textbf{Veteran's Day (no class)} & Simulation/probability \\
    \hline
    13 & 3:~Statistical inference & Monte Carlo integration/randomization tests & Lab \\
    \hline
    14 & 3:~Statistical inference & Re-sampling (Bootstrap/jackknife) & Lab \\
    \hline
    15 & 3:~Statistical inference & Maximum likelihood estimation (MLE) & Lab \\
    \hline
    16 & 3:~Statistical inference & Markov Chain Monte Carlo (MCMC) & Lab/\textbf{presentations} \\
    \hline
    \hline
    17 & All & & \textbf{Final exam} \\
    \hline
\end{tabular}
\end{center}

\end{document}