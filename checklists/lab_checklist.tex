\documentclass[12pt]{article}
\usepackage[margin=0.25in,letterpaper]{geometry}
\usepackage[ampersand]{easylist}

%% or use amsymb for open and closed items
%% \usepackage{amssymb}
%% \begin{itemize}
%% \item[$\square$] An open item.
%% \item[$\boxtimes$] A closed item.
%% \end{itemize}

\title{Lab Checklist for \\ \normalsize STAT 445/645 Intro to Statistical Computing}
\author{
        AG Schissler \\
                Department of Mathematics \& Statistics\\
        University of Nevada, Reno
}
\date{\today}



\begin{document}
\maketitle


Please use this checklist~\cite{Gawande2010} to ensure that your lab report is well prepared. I recommend adding and sharing your own ideas for this list.

\section*{Checklist}

\begin{easylist}[checklist]
& 1.~Completeness
&& Answered all major questions (e.g., \bf{1a}).
&& Answered all minor questions within questions (e.g., is that surprising?).
&& Attempted all challenges.
& 2.~Accuracy
&& Asked classmates about any questions I'm unsure of.
&& Asked the instructor any questions that we're unsure of.
&& Checked with classmate(s) on all answers and came to a consensus.
&& Answered all questions precisely (e.g. give your answer in Mb not bytes).
&& Double checked for bugs or conceptual errors after drafting the responses.
& 3.~Presentation
&& Used RMarkdown to ``fontify'' code, responses, and output.
&& Designed, labeled, and placed figures appropriately.
&& Organized responses near the question asked.
&& Edited code for readability and brevity.
&& Modified code so that it's output is concise and readable.
& 4.~Communication
&& Used appropriate language and vocabulary.
&& Commented code appropriately.
&& Answered the questions and only the question directly (no fluff).
&& Double checked spelling and grammar.
&& Edited my responses to be concise yet thoughtful.
\end{easylist}

\bibliographystyle{abbrv}

\begin{thebibliography}{10}

\bibitem{Gawande2010}
Gawande, Atul (2010) {\em The Checklist Manifesto}, vol.~1. Penguin Books

\end{thebibliography}

\end{document}
