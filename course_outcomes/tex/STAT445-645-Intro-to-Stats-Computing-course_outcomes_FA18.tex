%A simple LaTeX syllabus template by Luke Smith.
%I believe this is a variation of a template I got by Mike Hammond...?

%\documentclass[11pt,twocolumn]{article}
\documentclass[11pt,onecolumn]{article}
\usepackage{hyperref}
\usepackage[margin=0.75in]{geometry}
\usepackage{titlesec}
\usepackage{longtable}
\usepackage{gb4e}
%\usepackage[margin=.75in]{geometry}
\usepackage{textcomp}
\pagenumbering{gobble}
%% \usepackage[utf8]{inputenc}
%% \usepackage[english]{babel}
%% \usepackage[usenames, dvipsnames]{color}

%paragraph formatting
\setlength{\parindent}{0pt}
\setlength{\parskip}{3pt}

\usepackage{url}

\hypersetup{
    colorlinks=true,
    linkcolor=blue,
    filecolor=magenta,      
    urlcolor=blue,
}

\titlespacing\subsection{0in}{\parskip}{\parskip}
\titlespacing\section{0in}{\parskip}{\parskip}

%Just fill these in to fill in the basic syllabus information.
\newcommand{\coursename}{Introduction to Statistical Computing - STAT 445/645}
\newcommand{\semester}{Fall 2018}
\newcommand{\roomnumb}{DMSC 106}
\newcommand{\classtimes}{Mon,Wed 2:30pm - 3:45pm}
\newcommand{\myname}{A.~Grant Schissler}
%\newcommand{\myemail}{\href{mailto:aschissler@unr.edu}{aschissler@unr.edu}}
\newcommand{\myemail}{aschissler@unr.edu}
\newcommand{\office}{DMSC 224}
\newcommand{\officehours}{Tue 2:30pm-3:30pm, Wed 1:30pm-2:30pm, or by appointment}
% \newcommand{\university}{The University of Nevada, Reno}
\renewcommand{\familydefault}{\sfdefault}

\title{\textbf{\coursename}}
% \author{{\university}---{\semester}---{\roomnumb}---{\classtimes}}
\author{{\semester}---{\roomnumb}---{\classtimes}}
\date{}


\begin{document}
\maketitle

\vspace{-0.25in}
\noindent\makebox[\linewidth]{\rule{\textwidth}{1pt}}

\begin{center}
\begin{tabular}{llll}
\textbf{Instructor}:&\myname & \textbf{Contact}:&\href{mailto:\myemail}{\myemail}, 775-784-4661 (office)\\
\textbf{Office}:&\office & \textbf{Hours}:&\officehours\\
\end{tabular}
\end{center}

\section*{Catalog Description}
Introduction to statistical computing; data visualization and manipulation; document creation; graphics; simulation techniques; parallel computing; estimation; optimization; advanced statistical methods.

\section*{400-level Student Learning Outcomes}

\begin{description}
\item[UG1] Students will be able to implement statistical simulation, re-sampling techniques, and maximum likelihood estimation. 
\item[UG2] Students will be able to conduct a simulation-based power analysis. 
\item[UG3] Students will be able to write professional quality reports and computer code.
\item[GRAD1] Students will be able to use statistical computing methods to complete a research project and effectively communicate their findings.
\end{description}

\section*{Course outcomes}

Students will be able to $\ldots$

\begin{enumerate}
  \itemsep0em 
\item use R and RStudio.
\item produce HTML, PDF, or Word documents using R Markdown.
\item install R packages for computing tasks.
\item use R help functions.
\item use R vectors.
\item use R factors.
\item use R lists.
\item use \textsc{data.frames}.
\item control flow in R using conditionals, etc..
\item iterate using loops in R.
\item iterate using the \textsc{apply} family.
\item manipulate numeric and text data using base R utilities.
\item write functions in R.
\item benchmark/profile R code.
\item write parallel R code.
\item import data from flat files, including .csv, .txt, .xlsx.
\item clean data for data analysis.
\item conduct an exploratory data analysis.
\item visualize data.
\item implement the graphic of graphics.
\item generate pseudo-random numbers.
\item simulate data via Monte Carlo techniques.
\item conduct simulation-based hypothesis tests.
\item conduct a simulation-based power analysis.
\item integrate functions using Monte Carlo techniques.
\item use re-sampling for statistical inference (bootstrap, jackknife).
\item use maximum likelihood estimation for statistical inference.
\item use Markov Chain Monte Carlo (MCMC) to sample from probability distributions.
\end{enumerate}


\end{document}